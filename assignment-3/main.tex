
\let\negmedspace\undefined
\let\negthickspace\undefined
\documentclass[journal]{IEEEtran}
\usepackage[a5paper, margin=10mm, onecolumn]{geometry}
%\usepackage{lmodern} % Ensure lmodern is loaded for pdflatex
\usepackage{tfrupee} % Include tfrupee package

\setlength{\headheight}{1cm} % Set the height of the header box
\setlength{\headsep}{0mm}     % Set the distance between the header box and the top of the text

\usepackage{gvv-book}
\usepackage{gvv}
\usepackage{cite}
\usepackage{amsmath,amssymb,amsfonts,amsthm}
\usepackage{algorithmic}
\usepackage{graphicx}
\usepackage{textcomp}
\usepackage{xcolor}
\usepackage{txfonts}
\usepackage{listings}
\usepackage{enumitem}
\usepackage{mathtools}
\usepackage{gensymb}
\usepackage{comment}
\usepackage[breaklinks=true]{hyperref}
\usepackage{tkz-euclide} 
\usepackage{listings}
\usepackage{gvv}                                        
\def\inputGnumericTable{}                                 
\usepackage[latin1]{inputenc}                                
\usepackage{color}                                            
\usepackage{array}                                            
\usepackage{longtable}                                       
\usepackage{calc}                                             
\usepackage{multirow}                                         
\usepackage{hhline}                                           
\usepackage{ifthen}                                           
\usepackage{lscape}
\begin{document}

\bibliographystyle{IEEEtran}
\vspace{3cm}

\title{JEE-MAINS-29/01/2023-shift-1}
\author{AI24BTECH11024-Pappuri Prahladha}
% \bigskip
{\let\newpage\relax\maketitle}

\renewcommand{\thefigure}{\theenumi}
\renewcommand{\thetable}{\theenumi}
\setlength{\intextsep}{10pt} % Space between text and floats


\numberwithin{equation}{enumi}
\numberwithin{figure}{enumi}
\renewcommand{\thetable}{\theenumi}
\section{Section A.MCQs}
\begin{enumerate}[start=16]
\item Fifteen football players of club-team are given 15 T-shirts with their names written on the backside. If the players pick up the T-shirts randomly, then the probability that at least 3 players pick the correct T-shirt is
\begin{enumerate}
    \item $\frac{5}{24}$
    \item $\frac{2}{15}$
    \item $\frac{1}{6}$
    \item $\frac{5}{36}$
\end{enumerate}
\item Let \\$f\brak{\theta} = 3 \brak{ \sin^{4}\brak{ \frac{3\pi}{2} - \theta } + \sin^{4}\brak{ 3\pi + \theta } } - 2 \brak{ 1 - \sin^{2}\brak{2\theta} }$
\\and $S= \brak{ \theta \in \brak{0, \pi} : f'\brak{\theta} = -\frac{\sqrt{3}}{2} }$. If $ 4\beta = \sum_{\theta \in S} \theta $, then $ f\brak{\beta} $ is equal to 
\begin{enumerate}
    \item $\frac{11}{8}$
    \item $\frac{5}{4}$
    \item $\frac{9}{8}$
    \item $\frac{3}{2}$
\end{enumerate}
\item If p, q and r are the three propositions, then which of the following combinations of the truth values of p, q and r makes the logical expression 
$\cbrak{\brak{p \lor q}\land \brak{\brak{\sim p}\lor r}} \to \brak{\brak{\sim q} \lor r}$ false ?
\begin{enumerate}
    \item $p=T,q=F,r=T$
    \item $p=T,q=T,r=F$
    \item $p=F,q=T,r=F$
    \item $p=T,q=F,r=F$
\end{enumerate}
\item There rotten apples are mixed accidently withseven good apples and four apples are drawn oneby one without replacement. Let the random
variable X denote the number of rotten apples. If $\mu$and $\sigma^{2}$ represent mean and variance of X,respectively, $10\brak{\mu^{2}+\sigma^{2}}$ then  is equal to 
\begin{enumerate}
    \item 20
    \item 250
    \item 25
    \item 30
\end{enumerate}
\item Let $y=f\brak{x}$ be the solution of the differential equation $y\brak{x+1}dx-x^{2}dy=0,y\brak{1}=e$. The $\lim_{x\to0^{+}}f\brak{x}$ is equal to 

\begin{enumerate}
    \item 0
    \item $\frac{1}{e}$
    \item $e^{2}$
    \item $\frac{1}{e^{2}}$
\end{enumerate}
\section{Section B.Numerical}
\item Let the co-ordinats of one vertex of $\Delta ABC$ be $A\brak{0, 2, \alpha}$ and the other two vertices lie on the line $\frac{x+\alpha}{5}=\frac{y-1}{2}=\frac{z+4}{3}$. For $\alpha\in Z$, if the area of $\Delta ABC$ is 21 sq.units and the line segment BC has length $2\sqrt{21}$ units, then $\alpha^{2}$ is equal to \underline{\hspace{2.5cm}}.\\
\item Let the equation of the plane p containing the line $x+10=\frac{8-y}{2}=z$ be $ax+by+3z=2\brak{a+b}$ and the distance of the plane P from the point $\brak{1, 27, 7}$ be $c$. Then $a^{2}+b^{2}+c^{2}$ ii equal to \underline{\hspace{2.5cm}}.\\
\item Suppose f is a function satisfying $f\brak{x+y}=f\brak{x}+f\brak{y}$ for all $x,y \in N$ and $f\brak{1}=\frac{1}{5}$. If $\sum_{n=1}^{m}\frac{f\brak{n}}{n\brak{n+1}\brak{n+2}}=\frac{1}{12}$, then m is equal to \underline{\hspace{2.5cm}}.\\
\item Let $a_{1},a_{2},a_{3},\ldots$ be a GP of increasing positive numbers. If the product of fourth and sixth terms is 9 and the sum of fifth and seventh terms is 24, then $a_{1}a_{9}+a_{2}a_{4}a_{9}+a_{5}+a_{7}$ is equal to \underline{\hspace{2.5cm}}.\\
\item Let $\overrightarrow{a},\overrightarrow{b}$ and $\overrightarrow{c}$ be three non-coplanar vectors. Let the position vectors of four points $A, B, C and D$ be $\overrightarrow{a}-\overrightarrow{b}+\overrightarrow{c},\lambda\overrightarrow{a}-3\overrightarrow{b}+4\overrightarrow{c},-\overrightarrow{a}+2\overrightarrow{b}-3\overrightarrow{c}$ and $2\overrightarrow{a}-4\overrightarrow{b}+6\overrightarrow{c}$ respectively. If $\overrightarrow{AB},\overrightarrow{AC}$ and $\overrightarrow{AD}$ are coplanar, then $\lambda$ is $\colon$\\
\item If all the six digit numbers $X_{1}X_{2}X_{3}X_{4}X_{5}X_{6}$ with $0<X_{1}<X_{2}<X_{3}<X_{4}<X_{5}<X_{6}$ are arranged in the increasing order, then the sum of the digits in the $72^{th}$ number is \underline{\hspace{2.5cm}}.\\
\item Let $f\colon R \to R$ be a differentiable function that satisfies the relation $f\brak{x+y}=f\brak{x}+f\brak{y}-1$, $\forall{x},y\in R$. If $f^{\prime}\brak{0}=2$, then $\lvert f\brak{-2}\rvert$ is equal to \underline{\hspace{2.5cm}}.\\
\item If the co-efficient of $x^{9}$ in $\brak{\alpha x^{3}+\frac{1}{\beta x}}^{11}$ and the co-efficient of $x^{-9}$ in $\brak{\alpha x-\frac{1}{\beta x^{3}}}^{11}$ are equal, then $\brak{\alpha\beta}^{2}$ is equal to \underline{\hspace{2.5cm}}.\\
\item Let the coefficients of three consecutive terms in the binomial expansion of $\brak{1+2x}^{n}$ be in the ratio $2\colon5\colon8$. Then the codfficient of the term, which  is in the middle of these terms, is \underline{\hspace{2.5cm}}.\\
\item Five digit numbers are formed using the digits 1, 2, 3, 5,7 with repetitions and are written in descending order with serial numbers. For example, the number 77777 has serial number 1. Then the serial number of 35337 is \underline{\hspace{2.5cm}}.\\



























\end{enumerate}
\end{document}
